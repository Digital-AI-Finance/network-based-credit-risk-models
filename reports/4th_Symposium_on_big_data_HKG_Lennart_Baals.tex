\documentclass[9pt, aspectratio=169, compress]{beamer}
\usepackage[utf8]{inputenc}
\usepackage[T1]{fontenc}
\usepackage{graphicx}
\usepackage{adjustbox}
\usepackage{courier}
\usepackage{natbib}
\usepackage{listings}
\usepackage{amsmath}
\usepackage{amsfonts}
\usepackage{amssymb}
\usepackage{booktabs}
\usepackage{mathtools}
\usepackage{tikz}
\usepackage{enumitem}
\usepackage{caption}
\usepackage{multicol}
\usepackage{hyperref}

\usetheme{Warsaw}
\usecolortheme{default}
\useoutertheme{miniframes}

\definecolor{lightgray}{gray}{0.75} 

% Table of Contents at beginning of each section
\AtBeginSection[]{}

% Header and footer
\setbeamertemplate{footline}{%
  \leavevmode%
  \hbox{%
    \begin{beamercolorbox}[wd=.5\paperwidth,ht=1.5ex,dp=1.125ex,leftskip=.3cm plus1fill,rightskip=.3cm]{author in head/foot}%
      \usebeamerfont{author in head/foot}%
      \insertshortauthor
    \end{beamercolorbox}%
    \begin{beamercolorbox}[wd=.5\paperwidth,ht=1.5ex,dp=1.125ex,leftskip=.3cm,rightskip=.3cm plus1fil]{title in head/foot}%
      \usebeamerfont{title in head/foot}%
      \hfill\insertframenumber/\inserttotalframenumber%
    \end{beamercolorbox}%
  }%
}

\setbeamertemplate{frametitle}{
  \nointerlineskip
  \begin{beamercolorbox}[ht=0.8em,dp=0.8ex,wd=\paperwidth]{frametitle}
    \usebeamerfont{frametitle}\insertframetitle
  \end{beamercolorbox}
}

\title[A Systematic Literature Review on Graph-Based Models]{A Systematic Literature Review on Graph-Based Models in Credit Risk Assessment}
\author[Lennart J. Baals, Yiting Liu, Jörg Osterrieder, Branka Hadji-Misheva]{Lennart J. Baals, Yiting Liu, Jörg Osterrieder, Branka Hadji-Misheva \\
\vspace{0.25cm}
\small{Department of High-Tech Business and Entrepreneurship, Faculty of Industrial Engineering and Business Information Systems, University of Twente, Enschede, Netherlands} \\
\small{Bern Business School, Bern University of Applied Science, Bern, Switzerland}}
\institute{2024 4th International Symposium on Big Data and Artificial Intelligence, Hong Kong}
\date{\today}

\begin{document}

% Title Slide
\begin{frame}
  \titlepage
\end{frame}

% Table of Contents Slide
\begin{frame}{Overview}
  \tableofcontents
\end{frame}

% Introduction Section
\section{Introduction}

% Motivation Slide
\begin{frame}{Motivation for the Use of Graph-Based Models in Credit Risk Assessment}
\begin{itemize}
    \item \textbf{Current Limitations of Traditional Models:}
    \begin{itemize}
        \item Rely on static borrower-specific features (e.g., credit scores, financial ratios).
        \item Assume independence among borrowers, neglecting interconnectedness.
        \item Fail to address systemic risks and contagion in financial networks.
    \end{itemize}
    \item \textbf{Advantages of Graph-Based Models:}
    \begin{itemize}
        \item Capture latent relationships (e.g., social, economic, and transactional ties).
        \item Identify systemic vulnerabilities and propagation of financial distress.
        \item Enable dynamic analysis of evolving networks (e.g., P2P lending, interbank markets).
    \end{itemize}
    \item \textbf{Increasing Adoption in Finance:}
    \begin{itemize}
        \item Applications in P2P lending, SME credit scoring, and systemic risk management.
        \item Growing interest due to advancements in Graph Neural Networks (GNNs) and network analytics.
    \end{itemize}
\end{itemize}
\end{frame}

% Research Questions Slide
\begin{frame}{Research Questions}
\begin{itemize}
    \item \textbf{Primary Research Question:}
    \begin{itemize}
        \item How do different graph-based models compare in enhancing credit risk assessment?
    \end{itemize}
    \item \textbf{Sub-Questions:}
    \begin{itemize}
        \item What types of graph-based models (e.g., GNNs, centrality measures, community detection) are most commonly used?
        \item What specific applications (e.g., default prediction, systemic risk analysis) benefit from these models?
        \item What challenges and opportunities exist for future adoption in credit risk assessment?
    \end{itemize}
    \item \textbf{Broader Objective:}
    \begin{itemize}
        \item Bridge the gap between traditional risk assessment techniques and network-driven methodologies.
    \end{itemize}
\end{itemize}
\end{frame}

% Contributions Slide
\begin{frame}{Contributions of the Study}
\begin{itemize}
    \item \textbf{Systematic Literature Review (SLR):}
    \begin{itemize}
        \item Comprehensive review of a final selection of 78 scholarly articles from top databases (Scopus, Web of Science).
        \item Application of structured coding to identify trends, methods, and key insights.
    \end{itemize}
    \item \textbf{Mapping the Research Landscape:}
    \begin{itemize}
        \item Analysis of temporal trends (e.g., significant growth since 2018).
        \item Insights into journal coverage (e.g., focus on finance, computer science, and economics).
        \item Keyword analysis to highlight interdisciplinary themes.
    \end{itemize}
    \item \textbf{Insights into Applications:}
    \begin{itemize}
        \item Utility in diverse settings:
        \begin{itemize}
            \item P2P lending: Improved credit scoring and borrower profiling.
            \item SME credit scoring: Integration of transactional and relational data.
            \item Systemic risk: Understanding contagion in interbank networks.
        \end{itemize}
    \end{itemize}
    \item \textbf{Explore Future Directions:}
    \begin{itemize}
        \item Identified gaps in hybrid model integration and real-time credit scoring.
        \item Recommendations for expanding functionality (e.g., predictive modeling in decentralized finance).
    \end{itemize}
\end{itemize}
\end{frame}


% Methodology Section
\section{Methodology}

\begin{frame}{Systematic Review Process}
\textbf{Objective:} Conduct a comprehensive review of scholarly literature on graph-based models in credit risk assessment.
\vspace{0.4cm}

\textbf{Framework of \cite{varsha_p_s_how_2024}:}
\begin{itemize}
    \item 1. Define review goals by chosen research questions.
    \item 2. Develop a detailed review methodology to guide the approach.
    \item 3. Conduct an extensive literature search.
    \item 4. Apply inclusion and exclusion criteria.
    \item 5. Assess study quality and extract/code data.
    \item 6. Synthesize and analyze findings.
    \item 7. Report results systematically.
\end{itemize}
\vspace{0.4cm}

\textbf{Databases Used:} Scopus and Web of Science (broad coverage and robust search features).
\end{frame}


% Search and Inclusion Criteria Slide
\begin{frame}{Search and Inclusion Criteria}
\textbf{Search Query:} 
\begin{itemize}
    \item Keywords: "graph," "network models," "credit risk," "P2P lending," etc.
    \item Boolean Operators: AND/OR to refine results.
\end{itemize}
\vspace{0.3cm}
\textbf{Inclusion Criteria:}
\begin{itemize}
    \item Articles in scientific journals, written in English.
    \item Articles including predefined keywords in the title, abstract, or keywords.
\end{itemize}
\vspace{0.3cm}
\textbf{Exclusion Criteria:}
\begin{itemize}
    \item Publications before 1993, or from publishers like MDPI or Hindawi.
    \item Irrelevant abstracts, titles, or unavailable online.
\end{itemize}
\vspace{0.3cm}
\textbf{Outcome:}
\begin{itemize}
    \item From 1,066 retrieved articles, 78 were included after quality and relevance checks.
\end{itemize}
\end{frame}


\begin{frame}{Coding Framework}
\textbf{Purpose:}
\begin{itemize}
    \item Organize and analyze data systematically from reviewed articles.
    \item We applied a double-blind peer review process for the article coding.
\end{itemize}
\vspace{0.3cm}
\textbf{Criteria for Coding:}
\begin{itemize}
    \item Graph Type: Directed, undirected, weighted, etc.
    \item Application Situation: Credit scoring, systemic risk, etc.
    \item Research Questions: Goals of each study.
    \item Methodology: Types of graph-based models used.
    \item Data Source: P2P platforms, banking data, etc.
    \item Task Type: Classification, prediction, clustering, etc.
    \item Performance Metrics: AUC, F1-score, etc.
    \item Validation: Cross-validation, hold-out tests, etc.
\end{itemize}
\end{frame}


\section{Findings}

% Temporal Distribution Slide
\begin{frame}{Temporal Distribution of Literature}
\textbf{Key Insights:}
\begin{itemize}
    \item First study on graph-based credit risk assessment published in 2012 (Hu et al.).
    \item Significant growth in publications since 2018, with a peak in 2024.
\end{itemize}
\vspace{-0.3cm}
\begin{figure}[H]
    \centering
    \includegraphics[width=0.7\linewidth]{Graphics/Literature Review/Temporal distribution of literature.png} % Replace with actual figure file
    \caption{Temporal Distribution of Literature}
    \label{fig:temporal-distribution}
\end{figure}
\end{frame}

% Journal and Topic Distribution Slide
\begin{frame}{Journal and Topic Distribution}
\textbf{Key Insights:}
\begin{itemize}
    \item Top-cited papers are predominantly published in journals covering Business \& Economics, Computer Science, and Operations Research.
    \item Reflects interdisciplinary nature of the field.
    \item Significant presence in applied journals for practical methodologies.
\end{itemize}
\vspace{0.4cm}
\textbf{Visualization:}
\begin{itemize}
    \item Next slide.
\end{itemize}
\end{frame}

\begin{frame}{Journal and Topic Distribution}
\textbf{Key Insights:}
\begin{itemize}
    \item \textbf{Top Journals:}
    \begin{itemize}
        \item \textit{Journal of Financial Stability}, \textit{Expert Systems with Applications}, \textit{MIS Quarterly}.
    \end{itemize}
    \item \textbf{Top Topics:}
    \begin{itemize}
        \item Business \& Economics (45\%), Computer Science (30\%), Operations Research (15\%), Others (10\%).
    \end{itemize}
\end{itemize}
\vspace{0.3cm}

\begin{table}[H]
\centering
\small
\begin{tabular}{l l l}
\hline
\textbf{Journal}                   & \textbf{Focus Areas}                          & \textbf{Top-Cited Papers (\#)} \\ \hline
\textit{Journal of Financial Stability} & Business \& Economics                      & 2                              \\
\textit{Expert Systems with Applications} & Computer Science; Operations Research     & 4                              \\
\textit{MIS Quarterly}             & Computer Science; Business \& Economics      & 1                              \\
\textit{Applied Soft Computing}    & Computer Science                             & 2                              \\
\textit{Physica A}                 & Physics                                      & 1                              \\
Others (e.g., Omega, JME, EJOR)    & Interdisciplinary                            & 10                             \\ \hline
\end{tabular}
\caption{Summary of Top Journals and Topics}
\end{table}
\end{frame}


% Keyword Analysis Slide
\begin{frame}{Keyword Analysis}
\textbf{Key Insights:}
\begin{itemize}
    \item Most frequent keywords include "credit risk," "credit scoring," and "peer-to-peer lending."
    \item Emerging topics: "machine learning," "graph neural networks," and "financial stability."
\end{itemize}
\vspace{-0.8cm}
\begin{figure}[H]
    \centering
    \includegraphics[width=0.5\linewidth]{Graphics/Literature Review/Word Cloud.png} % Replace with actual figure file
    \caption{Keyword Analysis}
    \label{fig:keyword-analysis}
\end{figure}
\end{frame}

\begin{frame}{Application of Graph-based Models in P2P Lending and SME Credit Scoring}
\textbf{Key Insights:}
\begin{itemize}
    \item \textbf{P2P Lending:}
    \begin{itemize}
        \item Network-derived variables significantly improve default prediction \citep{giudici_network_2019, ahelegbey_latent_2019}.
        \item Studies show the role of borrower connectivity in influencing risk assessments \citep{liu_leveraging_2024}.
        \item Advanced techniques like latent factor models reveal clustering patterns \citep{ahelegbey_factorial_2019}.
    \end{itemize}
    \item \textbf{SME Credit Scoring:}
    \begin{itemize}
        \item Inter-firm relationships captured through transactional networks \citep{kou_bankruptcy_2021, vinciotti_effect_2019}.
        \item Enhanced credit scoring models using graph-based centrality measures \citep{giudici_network_2020, rishehchi_fayyaz_data-driven_2020}.
        \item Frameworks like GNNs outperform traditional scoring methods \citep{lee_graph_2021}.
    \end{itemize}
\end{itemize}
\end{frame}

\begin{frame}{Application of Graph-based Models in Systemic Risk Assessment in Banking}
\textbf{Key Insights:}
\begin{itemize}
    \item \textbf{Study Focus:}
    \begin{itemize}
        \item Interbank lending, cross-border credit networks, and systemic contagion \citep{poledna_multi-layer_2015}.
        \item Quantifying risk propagation across interconnected banking systems \citep{tonzer_cross-border_2015, cheng_regulating_2022}.
    \end{itemize}
    \item \textbf{Techniques Used:}
    \begin{itemize}
        \item Multi-layer networks to capture interbank dependencies \citep{poledna_multi-layer_2015}.
        \item Graphical Gaussian Models (GGMs) to model systemic risks \citep{cerchiello_conditional_2016}.
        \item Dynamic multi-layer networks for real-time risk monitoring \citep{jin_financial_2024}.
    \end{itemize}
    \item \textbf{Applications:}
    \begin{itemize}
        \item Identifying vulnerable nodes and key contagion pathways \citep{chen_network_2020}.
        \item Supporting macroprudential policy development for risk containment \citep{lin_roles_2022}.
    \end{itemize}
\end{itemize}
\end{frame}


\begin{frame}{Emerging Techniques: GNNs and Hypergraphs}
\textbf{Graph Neural Networks (GNNs):}
\begin{itemize}
    \item Combine node features with graph structure for advanced predictions \citep{liu_leveraging_2024}.
    \item Applications in credit risk prediction and fraud detection \citep{shi_enhancing_2023}.
    \item E.g., hierarchical GNNs capturing borrower-supplier relationships \citep{song_enhancing_2024}.
\end{itemize}
\vspace{0.3cm}
\textbf{Hypergraphs:}
\begin{itemize}
    \item Extend traditional graphs to model higher-order relationships \citep{shi_improved_2024}.
    \item Useful for capturing complex interactions in SME lending and P2P platforms.
    \item Show promise in systemic risk modeling with multi-entity transactions \citep{zhao_cross-border_2023}.
\end{itemize}
\end{frame}

\section{Challenges and Future Directions}

% Challenges Slide
\begin{frame}{Challenges in Graph-Based Credit Risk Assessment}
\textbf{Key Challenges:}
\begin{itemize}
    \item \textbf{Data Availability and Quality:}
    \begin{itemize}
        \item Limited access to detailed datasets, especially in real-world financial networks \citep{munoz-cancino_combination_2023}.
        \item Incomplete or noisy data impacts the robustness of graph-based models \citep{hu_network-based_2012}.
    \end{itemize}
    \item \textbf{Scalability and Computational Complexity:}
    \begin{itemize}
        \item High computational resources required for processing large and dynamic networks \citep{wu_cdgat_2023, yildirim_big_2021}.
        \item Difficulty in scaling models for real-time risk monitoring in volatile credit markets.
    \end{itemize}
    \item \textbf{Integration with Traditional Models:}
    \begin{itemize}
        \item Lack of established frameworks for combining graph-based and statistical methods \citep{dastile_statistical_2020}.
    \end{itemize}
\end{itemize}
\end{frame}



\begin{frame}{Future Directions in Graph-Based Credit Risk Assessment}
\begin{itemize}
    \item \textbf{Real-Time Credit Scoring:}
    \begin{itemize}
        \item Leverage graph-based models for dynamic monitoring of financial networks.
        \item Enhance predictive accuracy of these models in volatile credit markets.
    \end{itemize}
    \item \textbf{Decentralized Finance (DeFi) Applications:}
    \begin{itemize}
        \item Adapt graph models to assess credit risk in blockchain-based lending and DeFi platforms.
        \item Identify risks in smart contracts and lending pools using network patterns.
    \end{itemize}
    \item \textbf{Hybrid Frameworks for Risk Assessment:}
    \begin{itemize}
        \item Integrate graph-based methods with traditional models to create more robust systems.
        \item Develop frameworks for seamless adoption in financial institutions.
    \end{itemize}
\end{itemize}
\end{frame}


\begin{frame}{Conclusion}
\begin{itemize}
    \item \textbf{Summary of Key Findings:}
    \begin{itemize}
        \item Graph-based models enhance credit risk assessment by capturing complex interdependencies.
        \item Significant applications in P2P lending, SME credit scoring, and systemic risk assessment.
    \end{itemize}
    \item \textbf{Implications for Research and Practice:}
    \begin{itemize}
        \item Need for broader adoption of graph-based techniques in dynamic and decentralized contexts.
        \item Potential to improve financial stability and risk management in evolving markets.
    \end{itemize}
    \item \textbf{Limitations and Recommendations:}
    \begin{itemize}
        \item Challenges in data availability, scalability, and integration with traditional frameworks.
        \item Future research to focus on hybrid systems, real-time scoring, and applications in DeFi.
    \end{itemize}
\end{itemize}
\end{frame}

\begin{frame}{Acknowledgment}
\begin{itemize}
    \item We extend our sincere thanks to the \textit{Swiss National Science Foundation} for its financial backing of our project 100018E\_205487, which focuses on network-based credit risk models in P2P lending markets.
\end{itemize}
\end{frame}

\begin{frame}[allowframebreaks]{References}
\footnotesize
\bibliographystyle{apalike} % Choose your preferred bibliography style
\bibliography{references_SLR_2024} % Ensure your .bib file is named "references.bib"
\end{frame}

\end{document}
